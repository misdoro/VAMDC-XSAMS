\documentclass[a4paper]{article} \pagestyle{plain}
\setlength\parindent{0ex}
\usepackage{url}
\usepackage{graphicx}
\usepackage{amssymb,amsmath}






\renewcommand{\maketitle}{

\begin{titlepage}

\begin{center}
\includegraphics[width=14cm]{logos.png}
\end{center}

\sf{

\vspace*{2.0cm}
\noindent{
\huge{
\textbf{
The `Case-By-Case' Schema for Molecular States in XSAMS - v0.3
}}}
\vspace*{3.0cm}

}

\noindent \textbf{Document Information}

\vspace*{0.5cm}

\setlength{\parindent}{-1cm}

\noindent \begin{tabular}{p{1.7in}p{4.3in}} %\hline 
\textbf{Editors: } & C. Hill \\
\textbf{Authors:} & C. Hill \\ %\hline 
\textbf{Contributors:} & M.-L. Dubernet, J. Tennyson, E. Roueff \\ %\hline 
\textbf{Type of document:} & standards documentation \\ %\hline 
\textbf{Status:} & draft \\ %\hline 
\textbf{Distribution:} & public \\ %\hline 
\textbf{Work package:} & WP6 \\ %\hline 
\textbf{Version:} & 0.3rc \\ %\hline 
\textbf{Document code:} &  \\ %\hline 
\textbf{Directory and file name:} & \url{http://www.vamdc.org/documents/cbc_v0.2.pdf}  \\ %\hline 

\end{tabular}

\vspace*{2.0cm}

\noindent \begin{tabular}{p{1.7in}p{4.3in}}

\textbf{Abstract:} & 
This document describes the `case-by-case' XML Schema for molecular states within VAMDC-XSAMS.
\end{tabular}



\end{titlepage}

\noindent \textbf{Version History}

\textbf{}

\noindent \begin{tabular}{|l|l|l|l|} 
\hline 
\textbf{Version} & \textbf{Date} & \textbf{Modified By} & \textbf{Description of Change} \\ \hline 
0.1 & 16/12/2010 & C. Hill & first draft \\ \hline 
0.2 & 03/05/2011 & C. Hill & second draft \\ \hline
0.3rc & 17/06/2011 & C. Hill & third draft (release candidate) \\ \hline
\end{tabular}

\textbf{}

\noindent \textbf{Disclaimer}

\noindent The information in this document is subject to change without notice. Company or product names mentioned in this document may be trademarks or registered trademarks of their respective companies.


\textbf{}

\noindent \textbf{All rights reserved}

\noindent The document is proprietary of the VAMDC consortium members. No copying or distributing, in any form or by any means, is allowed without the prior written agreement of the owner of the property rights.

\noindent 

\noindent This document reflects only the authors' view. The European Community is not liable for any use that may be made of the information contained herein.

\textbf{}

\noindent \textbf{Acknowledgements}

\noindent VAMDC is funded under the ``Combination of Collaborative Projects and Coordination and  Support Actions'' Funding Scheme of The Seventh Framework Program. Call topic: INFRA-2008-1.2.2 Scientific Data Infrastructure. Grant Agreement number: 239108.

\textbf{}

}









\begin{document}

\maketitle
\pagebreak

\section*{Introduction}

The `case-by-case' XML description of molecular states within XSAMS is designed to provide a straightforward and `flat' data structure for representing the quantum numbers and symmetries that define a molecular state. As of version 0.2.1, 12 cases have been implemented, covering the needs of the HITRAN, CDMS, and BASECOL databases.

Each case is identified by a \texttt{prefix} and a \texttt{version}, and belongs to the namespace given (currently) by the URI \url{http://www.ucl.ac.uk/~ucapch0/XSAMS/cases/<prefix>/<version>}. At least for the time being, it is suggeested that validation is through Namespace Validation Dispatch Language (NVDL). This may be implemented by including the relevant processing instruction in the XSAMS XML instance document, after the XML declaration. For example, using the oxygen editor
\begin{verbatim}
<?xml version="1.0" encoding="UTF-8"?>
<?oxygen NVDLSchema="cbc.nvdl"?>
...
\end{verbatim}
The NVDL Schema document, \texttt{cbc.nvdl} contains a set of rules which link the namespaces encountered in the XML document with the Schemata required to validate them. An sample NVDL document may be downloaded from \url{http://www.ucl.ac.uk/~ucapch0/XSAMS/}.

\section*{The Cases}

The identified cases are described in the following section; some examples are given below. It should be noted that the rovibronic states of different electronic states of a molecule may be described using different cases. For example, the ground, $X^2\Pi$ electronic state of $\mathrm{NO}$ may be described within the \texttt{hunda} case whereas the excited, $A^2\Sigma^+$ electronic state would be better described using the \texttt{hundb} case. The electronic state is identified by its single-letter spectroscopic symbol ($X$, $A$, $a$, $B$, etc.)

There follows a list of the cases identified in version 0.2.1 of XSAMS with some examples:

\begin{enumerate}
\item[0.] General case for arbitrary quantum numbers and symmetry (\texttt{gen})
\item Diatomic closed shell (\texttt{dcs}): CO, $\mathrm{N_2}$, $\mathrm{NO^+}$
\item Hund's case (a) diatomics (\texttt{hunda}): NO, OH [for low $J$]
\item Hund's case (b) diatomics (\texttt{hundb}): $\mathrm{O_2}$, OH [for high $J$]
\item Closed-shell, linear triatomic molecules (\texttt{ltcs}): $\mathrm{CO_2}$, HCN
\item Closed-shell, non-linear triatomic molecules (\texttt{nltcs}): $\mathrm{H_2O}$
\item Closed-shell, symmetric top molecules (\texttt{stcs}): $\mathrm{NH_3}$, $\mathrm{CH_3Cl}$
\item Closed-shell, linear, polyatomic molecules (\texttt{lpcs}): $\mathrm{C_2H_2}$
\item Closed-shell, asymmetric top molecules (\texttt{asymcs}): $\mathrm{C_2H_4}$
\item Open-shell, asymmetric top molecules (\texttt{asymos}): $\mathrm{CH_3O}$
\item Closed-shell, spherical top molecules (\texttt{sphcs}): $\mathrm{CH_4}$, $\mathrm{SF_6}$
\item Open-shell, spherical top molecules (\texttt{sphos})
\item Open-shell, linear triatomic molecules (\texttt{ltos}): $\mathrm{CCH}$
\item Open-shell, linear, polyatomic molecules (\texttt{lpos}): $\mathrm{C_3H}$, $\mathrm{C_{10}H}$
\item Open-shell, non-linear triatomic molecules (\texttt{nltos}): $\mathrm{HO_2}$, $\mathrm{CH_2}$
\end{enumerate}

\section*{Examples}
The element \texttt{<case:QNs>} should be placed within the XSAMS element \texttt{MolecularState} (although this cannot be enforced using NVDL). Some examples follow.

\subsection*{Example 1: a rovibrational state of CO}

The $v=0, J=1$ state of the ground electronic state of CO has the following representation in the `case-by-case' formulism:

\begin{verbatim}
<dcs:QNs>
   <dcs:ElecStateLabel>X</dcs:ElecStateLabel>
   <dcs:v>1</dcs:v>
   <dcs:J>0</dcs:J>
</dcs:QNs>
\end{verbatim}

\subsection*{Example 2: a rovibrational state of $\mathrm{NH_3}$}

The $J=22, K=10$ rotational state of the $(1,0^+,0^0,2^2)$ vibrational level of $\mathrm{NH_3}$ could be represented by the following XML:

\begin{verbatim}
<stcs:QNs>
   <stcs:ElecStateLabel>X</stcs:ElecStateLabel>
   <stcs:vi mode="1">1</stcs:vi>
   <stcs:vi mode="2">0</stcs:vi>
   <stcs:vi mode="3">0</stcs:vi>
   <stcs:vi mode="4">2</stcs:vi>
   <stcs:li mode="3">0</stcs:li>
   <stcs:li mode="4">2</stcs:li>
   <stcs:vibInv>s</stcs:vibInv>
   <stcs:vibSym>E</stcs:vibSym>
   <stcs:J>20</stcs:J>
   <stcs:K>10</stcs:K>
</stcs:QNs>
\end{verbatim}

\subsection*{Example 3: two states of different electronic states of MgH}

This example shows two states of the MgH radical, using different cases for the $X^2\Sigma^+$ and $A^2\Pi$ electronic states:

\begin{verbatim}
<MolecularState stateID="S1-MgH-1">
  <Description>
    A state in the ground electronic state, X(2Sigma+), of MgH
  </Description>
  <MolecularStateCharacterisation>
    <StateEnergy energyOrigin="Zero-point from calculation based on N^2 Hamiltonian">
      <Value units="1/cm">0.</Value>
    </StateEnergy>
    <TotalStatisticalWeight>4</TotalStatisticalWeight>
  </MolecularStateCharacterisation>
  <hundb:QNs>
    <hundb:ElecStateLabel>X</hundb:ElecStateLabel>
    <hundb:Lambda>0</hundb:Lambda>
    <hundb:S>0.5</hundb:S>
    <hundb:v>0</hundb:v>
    <hundb:J>0.5</hundb:J>
    <hundb:N>0</hundb:N>
    <hundb:SpinComponentLabel>1</hundb:SpinComponentLabel>
    <hundb:parity>+</hundb:parity>
    <hundb:kronigParity>e</hundb:kronigParity>
  </hundb:QNs>
</MolecularState>
  ...
<MolecularState stateID="S1001-MgH-1">
  <Description>
    A state in the first excited electronic state, A(2Pi) of MgH
  </Description>
  <MolecularStateCharacterisation>
    <StateEnergy energyOrigin="Zero-point of electronic ground state">
      <Value units="1/cm">19273.2694</Value>
    </StateEnergy>
    <TotalStatisticalWeight>4</TotalStatisticalWeight>
  </MolecularStateCharacterisation>
  <hunda:QNs>
    <hunda:ElecStateLabel>A</hunda:ElecStateLabel>
    <hunda:Lambda>1</hunda:Lambda>
    <hunda:Omega>0.5</hunda:Omega>
    <hunda:S>0.5</hunda:S>
    <hunda:v>0</hunda:v>
    <hunda:J>0.5</hunda:J>
    <hunda:parity>-</hunda:parity>
    <hunda:kronigParity>f</hunda:kronigParity>
  </hunda:QNs>
</MolecularState>
\end{verbatim}

% case-by-case descriptions
%%%

\input{allcases-0.3.tex}

\end{document}
